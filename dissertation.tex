
% book example for classicthesis.sty
\documentclass[
  % Replace twoside with oneside if you are printing your thesis on a single side
  % of the paper, or for viewing on screen.
  oneside,
  %twoside,
  11pt, a4paper,
  footinclude=true,
  headinclude=true,
  cleardoublepage=empty
]{scrbook}

\usepackage{lipsum}
\usepackage[linedheaders,parts,pdfspacing]{classicthesis}
\usepackage{amsmath}
\usepackage{amsthm}
\usepackage{mathtools}
\usepackage{acronym}
\usepackage{dissertation}
\usepackage{multirow}
\usepackage{pdfpages}
\usepackage{float}
\usepackage{svg}
%\usepackage[acronym]{glossaries}

% Title
\title{Advanced Light Transport Algorithms on Heterogeneous Platforms: evaluation of the DICE framework}

% Author
\author{César Morais Perdigão}

% Supervisor
\def\supervisor{%
	Luís Paulo Peixoto dos Santos
\\
	Alberto José G. de C. Proença
}

% Date
\date{\today}

%%Defines
%\def \... {..}

\makeglossaries  %  either use this ...

%\makeindex	% ... or this

\newcommand*{\myglossaryindent}{0.65cm}         
\newcommand*{\myglsdescwidth}{10cm}                     
\newglossarystyle{altlong4colwithindent}
{
  \glossarystyle{altlong4col}
  \renewenvironment{theglossary}
    {\begin{tabular}[l]{@{\hspace{\myglossaryindent}}lp{\myglsdescwidth}lp{\glspagelistwidth}@{}}}
    {\end{tabular}}
}

\begin{document}
	
% Add acronym definitions
\newacronym{vcm}{VCM}{Vertex Connection and Merging}
\newacronym{mlt}{MLT}{Metropolis Light Transport}
\newacronym{pt}{PT}{Path Tracing}
\newacronym{bdpt}{BDPT}{Bidirectional Path Tracing}
\newacronym{pm}{PM}{Photon Mapping}
\newacronym{ppm}{PPM}{Progressive Photon Mapping}
\newacronym{bpm}{BPM}{Bidirectional Photon Mapping}

\newacronym{mis}{MIS}{Multiple Importance Sampling}

\newacronym{cpu}{CPU}{Central Processing Unit}
\newacronym{gpu}{GPU}{Graphics Processing Unit}

\newacronym{mcmc}{MCMC}{Markov Chain Monte Carlo}

\newacronym{mic}{MIC}{Many Integrated Core}

\newacronym{sse}{SSE}{Streaming SIMD Extensions}
\newacronym{avx}{AVX}{Advanced Vextor Extensions}
\newacronym{simd}{SIMD}{Single Instruction Multiple Data}

\newacronym{rmse}{RMSE}{Root Mean Squared Error}
    
% Cover page ---------------------------------------------
\sf
	\pagenumbering{alph}
	\thispagestyle{empty}
	\input{def/title}
\rm
	\cleardoublepage
%---------------------------------------------------------
% Add acknowledgements

\chapter*{Acknowledgements}
%Write acknowledgements here

	\cleardoublepage
	
% Add abstracts (en,pt) -----------------------------------------------------------
\chapter*{Abstract}
Physically based rendering is an unsolved challenge, as there are many efficient light transport algorithms, although none is robust enough to handle every possible situation, lighting and scene configuration. The performance of such techniques can be enhanced using two different approaches: more efficient and robust algorithms and the use of more computational power. Although most modern computers have both powerful CPUs and GPUs, most current implementations of light transport algorithms only take advantage of one of these processors. Implementing new light transport algorithms such as Vertex Connection and Merging on these hybrid platforms, using both CPU and GPU, may provide more efficient results, although the suitability of such algorithms for heterogeneous computing platforms has not been tested yet, and that is the main goal of this project.

	\cleardoublepage

\chapter*{Resumo}
A sintese de imagens fotorrealistas é um desafio ainda por resolver, dado que existem muitos algoritmos de transporte de luz eficientes, porem nenhum lida com todas as situações de forma robusta. O desempenho destas técnicas pode ser melhorado usando algoritmos de transporte de luz mais eficientes e robustos bem como usando mais poder computacional. Apesar da maioria dos computadores modernos ter disponível poderosos CPU's e GPU's, a maioria das implementações utiliza apenas um destes processadores. Implementar novos algoritmos de transporte de luz tal como o Vertex Connection and Merging nestas plataformas hibridas pode trazer ganhos de desempenho, porem nenhum teste para avaliar a adequabilidade destes algoritmos para sistemas heterogéneos foi efetuado, e é esse o principal objectivo deste projeto.

	\cleardoublepage
	
	\pagenumbering{roman}
	\setcounter{page}{3}
	%pagestyle{fancy}   % -------- removed
	\rm
	
	% Document
	\cleardoublepage
    \phantomsection
    \addcontentsline{toc}{chapter}{Contents}
	\tableofcontents
	
	\cleardoublepage
	\listoffigures
	
	\cleardoublepage
	\listoftables
	
	%\cleardoublepage
	%\lstlistoflistings
	
	% Add list of acronyms
	\printglossary[type=\acronymtype, title=List of Acronyms, nonumberlist=true, toctitle=List of Acronyms, style=altlong4colwithindent]
	\cleardoublepage
	\pagenumbering{arabic}
	\setcounter{page}{3}

\part{Introductory material}

\chapter{Introduction}

One of the main challenges in computer graphics is physically based rendering, the synthetic creation of images that are indistinguishable from the perception of the real world based on a geometric description of the scene, materials and light sources. There are several algorithms that try to solve this problem, although none of them is yet robust enough to handle every possible situation.

One of the first solutions, \gls{pt} proposed by \cite{Kajiya}, aims to solve this problem by tracing ligh transport path starting from the camera until it hits a light source.

One improvement upon this algorithm was \gls{bdpt}, developed independently by \cite{Lafortune} and \cite{Veach}. Although with different mathematical background, the goal is to sample more light transport paths by connecting sub-paths generated from the camera and the light source. This allows for a much more efficient rendering of effects like caustics, although
effects like reflected caustics are still too difficult for a bidirectional path tracer to handle robustly.

In an effort to improve the efficiency and robustness of light transport algorithms, \cite{Veach} proposed the adaptation of the metropolis sampling algorithm to the light transport problem. The metropolis algorithm consists in starting from any point in the function domain, we apply mutations to this point with a carefully chosen acceptance probability, and the sampling pattern will be proportional to the value of the function. In the case of rendering, the image function is the estimated incident radiance value for each pixel and the integration domain is the set of all light transport paths.

One completely different approach developed by \cite{Jensen} was instead of trying to find paths from the light source to the camera to just trace a packets of photons throughout the scene and store them in an acceleration structure. In a second pass, the rays would start from the camera and consult the photon map in the vicinity of the intersections and calculate the expected radiance through a density estimation. Unlike all the previously presented methods, photon mapping introduces bias, that is, it may not converge to the correct result of the rendering equation. Although, it is consistent, and by diminishing the search radius on the photon map and increasing the number of traced photons, the bias reduces to zero in the limit \citep{Hachisuka}.

Most recently, an attempt to combine these two approaches was proposed by \cite{Georgiev}. In his algorithm, \gls{vcm}, photon mapping and bidirectional path tracing are combined, taking advantage of each of the algorithms strong points: the high convergence rate from bidirectional path tracing and the better handling of caustics from photon mapping. These two algorithms are combined by reducing photon mapping to a path sampling technique in the path integration space formulation and combines it with bidirectional path tracing using multiple importance sampling.

Despite the different aproaches usen in different techniques, what they all have in common is the need for computaional power, since all of these algorithms are based on ray tracing, a computationally expensive operation.

Although most computers nowadays contain both a \gls{cpu} and a \gls{gpu}, most commonly seen implementations only use one of there processors, and so wasting usefull computational power. Developing for this kind of hybrid platforms is not easy though, as it involves maintaining different implementations for each processor as well as dealing with separate memory addressing spaces. In order to solve these issues there is the framework DICE, developed in the University of Texas at Austin. This framework allows an easy workload distribution as well as a memory management system. Mapping light transport algorithms to these heterogeneous systems may lead to performance improvements although no test to evaluate the suitability of these algorithms has been conducted yet. 

With all this, the main goals for this project are to eavluate the relative performance and robustness of the light transport algorithms described previously, evaluate their suitability for heterogeneous systems and evaluate the DICE framework in terms of performance, usability and the possibilities provided.
\chapter{State of the Art}

\section{Path Tracing}

One of the first solution to the global ilumination was introduced by \cite{Kajiya} when he defined the problem of rendering as the resolution of an integral equation, also know as the rendering equation.

\begin{equation}
I(x,x')=g(x,x')[e(x,x')+\int_{S}^{} \rho(x,x',x'')I(x',x'')dx'']
\label{eq:render_eq}
\end{equation}

Where:

\begin{tabular}{r l}
$I(x,x')$ & is related to the intensity of light passing from point $x$ to $x'$ \\
$g(x,x')$ & is the geometric term \\
$e(x,x')$ & is related to the intensity of emited light from $x$ to $x'$ \\
$\rho(x,x',x'')$ & is related to the intensity of light scattered from $x''$ to $x$ through $x'$\\

\end{tabular}



\section{Bidirectional Path Tracing}






\section{Metropolis Light Trasport}





\section{Photon Mapping}
\subsection{Progressive Photon Mapping}




\section{Vertex Connection and Merging}




\section{DICE}
\chapter{The Problem and its Challenges}

%In this project, the main goals are to evaluate the relative performance and robustness of different light transport algorithms, as well as to evaluate their suitability to heterogeneous systems and evaluate the DICE framework in terms of performance and usability. To this effect, different implementations of the previously mentioned light transport algorithms will be developed for both \gls{cpu} and \gls{gpu} using specialized ray tracing frameworks targeted to the different devices: Optix targeted for \gls{gpu} and Embree for \gls{cpu}. The performance of the algorithms will be evaluated by comparing the image quality for a given rendering time and by measuring the rendering time needed to achieve a given image quality. In addition to this, the overhead of using the DICE framework will be measured and compared to the gain of using various devices. In a first stage of developent, all the algorithms will be developed independently for \gls{cpu} and \gls{gpu}. Then in a second stage, these implementations will be integrated in DICE for use in heterogenous systems. Finally all the performance measurements will be compared in order to evaluate all the previous points.

%As mentioned before, communication between different devices hinders performance, and although most algorithms require no synchronization, photon mapping requires sharing the photon map across all devices, or else there would be different variance through the image, which is a perceptible flaw. As \gls{vcm} generates a new photon map in each iteration, this photon map must be copied to all used devices in every iteration, which may negatively affect performance. Another challenge is the usage of different data structures by the different ray tracing frameworks which adds complexity to the development process.

%The unified memory address space abstraction supported by the DICE framework will be assessed in terms of how it handles such data requirements: shared data structures and diverse data layouts. The usability of the DICE memory model with regard to supporting the specific requirements to these lighting algorithms will be qualitatively assessed through the additional effort required from the programmer.




\part{Core of the dissertation}

\chapter{Experimental Results}

\section{Experimental Setup}

All experiments were executed on a dual-socket computer with two 10 core Intel Xeon E5-2670 v2 at the frequency of 2.50GHz, 64 GB of RAM and a Nvidia Tesla K20 GPU.

Every software used was updated to the versions listed in table~\ref{tab:soft_ver}.

\begin{table}[h]
\centering
\begin{tabular}{|l|l|}

\hline
Software & Version \\
\hline
Linux & 2.6.32-279 \\
\hline
GCC & 4.8.2 \\
\hline
CUDA Toolkit & 5.5 \\
\hline
Optix & 3.7 \\
\hline
Embree & 2.5.1 \\
\hline

\end{tabular}
\caption{\label{tab:soft_ver} Software Versions Used}
\end{table}

In order to test the scalability and efficiency of the studied algorithms, the execution time was measured using only the \gls{gpu}, and while using only the \gls{cpu} and using both devices with a varying number of \gls{cpu} cores. The warkload distribution between the devices was also measured by the DICE scheduler. Every time measurement was executed five times, selecting the best time as the final result. These tests were executed using the Living Room Scene with a resolution of 1027x768.

In order to evaluate the image quality produced by the studied algorithms, three scenes were used for testing, each with a different goal. The Sponza scene, courtesy of Crytek, is an outdoor scene with only diffuse materials, but complex geometry nonetheless. The Kitchen scene, courtesy of \todo[inline]{who to give credit to}, is an indoor scene with a lot of glossy materials. The final and most complex scene, the Living Room, courtesy of Iliyan Georgiev, is an indoor scene as well but with an emphasis on reflected caustics and complex lighting. For every scene the reference image was rendered with VCM with 100000 samples per pixel. Then for every algorithm a set of images with an approximate rendering time were produced. All these images were compared to the reference image using the \gls{rmse} metric.

\missingfigure{Imagens de referencia}


\chapter{Conclusions}

%All algorithms suitable to heterogeneous systems containing \gls{cpu}s and \gls{gpu}s. BPM has worst efficiency.

%Bidirectional algorithms penalized in outdoor scenes.

%VCM and BPM overhead when reflected caustics not present.

%Only a small set of DICE features used.

%DICE has full C++ generic interface. Easy enough to use and learn.

%Scheduler learing robust to handle workload distribution and maximizes efficiency.

%Future work: try to adapt code to multi-\gls{gpu} and develop for Xeon Phi.

This work main goal was to evaluate the suitability of a set of state of the art light transport algorithms, based on path space integration, for execution on parallel heterogeneous systems using the DICE framework. Simultaneously, we were interested on identifying DICE's strong and weak points, regarding both performance and usability.

Four path space integration algorithms were selected (path tracing, bidirectional path tracing, bidirectional photon mapping and vertex connection and merge) based on their relevance and robustness for solving the light transport problem. Robust and efficient kernels were developed for each of these algorithms using specialized and highly optimized ray tracing libraries, namely Optix for the \gls{gpu} and Embree for the \gls{cpu}. DICE was then used to manage execution on a heterogeneous platform with 20 \gls{cpu} cores and a \gls{gpu}.

Results demonstrated that all algorithms are able to efficiently exploit the available set of heterogeneous devices.  In fact, even when using all the devices, heterogeneous efficiency is well above 85\% for all algorithms. Execution times also show that the algorithms scale well within the evaluated heterogeneous system. No inflection point (i.e., a number of devices above which execution times starts to increase) is reached.

DICE is a complete framework for programming heterogeneous systems, however during this work, only a small set of the available features were used. As specialized implementations for each device were used, and memory transfers were inexistent, only the DICE scheduler was used. The scheduler was efficient and distributed workload in a way that maximized performance and efficiency. From an usability point of view, DICE uses a solid and simple C++ generic interface, making it and easy to program and interact with the existing rendering code, although there was the need to reconfigure some settings, namely the way DICE views the various available devices. Being able to consider the \gls{cpu} as a monolithic device or as a set of devices (\gls{cpu} cores) allows more sophisticated work decomposition schemes at the cost of developing simplicity. This configuration possibility was essential in photon mapping based algorithms as it allows a hybrid work decomposition, thus avoiding performance issues regarding communication.

With respect to the light transport algorithms, image quality measurements show that different algorithms are more appropriate (i.e., converge faster) for different scene characteristics. \gls{vcm}, being the most complete algorithm, is the more robust, always converging to the expected solution; \gls{vcm} excels when specular-diffuse-specular paths are present, which have zero or very low probability of being sampled by the remaining algorithms.

\section{Future Work}

Current results do not exhaustively evaluate scalability on a heterogeneous system, because one single \gls{gpu} is used. Extending the code to a multi-\gls{gpu} environment is a priority, even though not trivial due to limitations of the Optix framework. Another interesting possibility is to adapt the developed code to the Intel \gls{mic} architecture using Embree as well.

A very interesting line of research would be to understand and define metrics that characterize performance on an intuitive manner on heterogeneous systems. In fact, metrics such as speedup, efficiency, iso-efficiency, among others, are well understood for homogeneous systems, both for the fixed size and fixed time cases. There is clearly a need to define metrics for heterogeneous systems, which are well accepted and understood by the community. 

Lastly, a comparative study between DICE and StarPU in terms of performance and usability seems important in order to determine which is the best heterogeneous system development framework.

More sophisticated rendering algorithms, such as those based on Metropolis light transport combined with \gls{vcm}, have to be assessed with respect to both performance on heterogeneous systems and rate of convergence.




		
\bookmarksetup{startatroot} % Ends last part.
\addtocontents{toc}{\bigskip} % Making the table of contents look good.
\cleardoublepage

%----------------- Bibliography (needs bibtex) --------------------------------%
\bibliography{dissertation}
%----------------- Index of terms (needs  makeindex) --------------------------%
\printindex
%------------------------------------------------------------------------------%
	
	\appendix
	\renewcommand\chaptername{Appendix}

	% Add appendix chapters

%\part{Apendices}


	
\end{document}


\chapter{Conclusions and Future Work}

%All algorithms suitable to heterogeneous systems containing CPUs and GPUs. BPM has worst efficiency.

%Bidirectional algorithms penalized in outdoor scenes.

%VCM and BPM overhead when reflected caustics not present.

%Only a small set of DICE features used.

%DICE has full C++ generic interface. Easy enough to use and learn.

%Scheduler learing robust to handle workload distribution and maximizes efficiency.

%Future work: try to adapt code to multi-GPU and develop for Xeon Phi.

From the observed results, it is possible to infer that all sthe studied algorithms are suited for heterogeneous systems containing both \gls{cpu} and \gls{gpu}, as all algorithms use the available resources efficiently.

From the studied algorithms, \gls{vcm} is the more robust, allways converging to the solution expected, although with a slight overhead when not dealing with complex lighting.

DICE is a complete framework for programming heterogeneous systems, however during this work, only a small set of the available features were used. As specialized implementations for each device were used, and memory transfers were inexistent, only the DICE scheduler was used. The scheduler was efficient and distributed workload in a way that maximized performance and efficiency. From an usability point of view, DICE uses a simple C++ generic interface, making it and easy to program and interact with the existing rendering code, although there was the need to reconfigure some settings, namely the way DICE views the various available devices. Being able to consider the \gls{cpu} as a monolithic device or as a set of devices (\gls{cpu} cores) allows more sophisticated work decomposition schemes at the cost of developing simplicity. This configuration possiblity was essential in photon mapping based algorithms as it allows a hybrid work decomposition, thus avoiding performance issues.

As for future work, one of the main goals is to make the rendering code suitable for a multi-GPU environment, which due to limitations of the Optix framework is not trivial. Another interesting possibility is to adapt the developed code to the Intel \gls{mic} architecture using Embree as well. The study of other algorithms such as Metropolis Light Transport seems like an interesting possibility as well. Lastly, a comparative study between DICE and StarPU in terms of performance and usability seems important in order to determine which is the best heterogeneous system development framework.
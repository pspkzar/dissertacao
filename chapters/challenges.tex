\chapter{The Problem and its Challenges}

In this project, the main goals are to evaluate the relative performance and robustness of different light transport algorithms, as well as to evaluate their suitability to heterogeneous systems and evaluate the DICE framework in terms of performance and usability. To this effect, different implementations of the previously mentioned light transport algorithms will be developed for both \gls{cpu} and \gls{gpu} using specialized ray tracing frameworks targeted to the different devices: Optix targeted for \gls{gpu} and Embree for \gls{cpu}. The performance of the algorithms will be evaluated by comparing the image quality for a given rendering time and by measuring the rendering time needed to achieve a given image quality. In addition to this, the overhead of using the DICE framework will be measured and compared to the gain of using various devices. In a first stage of developent, all the algorithms will be developed independently for \gls{cpu} and \gls{gpu}. Then in a second stage, these implementations will be integrated in DICE for use in heterogenous systems. Finally all the performance measurements will be compared in order to evaluate all the previous points.

As mentioned before, communication between different devices hinders performance, and although most algorithms require no synchronization, photon mapping requires sharing the photon map across all devices, or else there would be different variance through the image, which is a perceptible flaw. As \gls{vcm} generates a new photon map in each iteration, this photon map must be copied to all used devices in every iteration, which may negatively affect performance. Another challenge is the usage of different data structures by the different ray tracing frameworks which adds complexity to the development process.

The unified memory address space abstraction supported by the DICE framework will be assessed in terms of how it handles such data requirements: shared data structures and diverse data layouts. The usability of the DICE memory model with regard to supporting the specific requirements to these lighting algorithms will be qualitatively assessed through the additional effort required from the programmer.

\section{Work Plan}

%This project will be devoloped in several stages. The first one was the literature review that concluded in the end of December 2014. The month of January was dedicated to the writing of this document and the implementation of the base structures needed for the algorithms studied. The next months of February and March will be dedicated to implement all the studied algorithms independently on the CPU and the GPU. In April, all the implementations will be integrated with DICE. The month of May will be dedicated to tests and measurements of performance and image quality of the previously developed implementations. Finaly in June all the work will be condensed in the writing of the PhD thesis.

In table ~\ref{tab:work_plan} is the current work plan for the development of the PhD thesis.

\begin{table}[h]
\centering
\begin{tabular}{|l|l|}
\hline
\textbf{Month} & \textbf{Work Phase} \\ \hline
October & \multirow{3}{*}{Literature Review} \\ \cline{1-1}
November &  \\ \cline{1-1}
December &  \\ \hline
January & Pre-dissertation Report Writing \\ \hline
February & \multirow{2}{*}{Homogeneous Implementations of Light Transport Algorithms on both CPU and GPU} \\ \cline{1-1}
March &  \\ \hline
April & Integration with DICE \\ \hline
May & Result Measurement and Evaluation \\ \hline
June & PhD Thesis Writing \\ \hline
\end{tabular}
\caption{\label{tab:work_plan} Work Plan}
\end{table}
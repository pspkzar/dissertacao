\chapter{The Problem and its Challenges}

%In this project, the main goals are to evaluate the relative performance and robustness of different light transport algorithms, as well as to evaluate their suitability to heterogeneous systems and evaluate the DICE framework in terms of performance and usability. To this effect, different implementations of the previously mentioned light transport algorithms will be developed for both \gls{cpu} and \gls{gpu} using specialized ray tracing frameworks targeted to the different devices: Optix targeted for \gls{gpu} and Embree for \gls{cpu}. The performance of the algorithms will be evaluated by comparing the image quality for a given rendering time and by measuring the rendering time needed to achieve a given image quality. In addition to this, the overhead of using the DICE framework will be measured and compared to the gain of using various devices. In a first stage of developent, all the algorithms will be developed independently for \gls{cpu} and \gls{gpu}. Then in a second stage, these implementations will be integrated in DICE for use in heterogenous systems. Finally all the performance measurements will be compared in order to evaluate all the previous points.

%As mentioned before, communication between different devices hinders performance, and although most algorithms require no synchronization, photon mapping requires sharing the photon map across all devices, or else there would be different variance through the image, which is a perceptible flaw. As \gls{vcm} generates a new photon map in each iteration, this photon map must be copied to all used devices in every iteration, which may negatively affect performance. Another challenge is the usage of different data structures by the different ray tracing frameworks which adds complexity to the development process.

%The unified memory address space abstraction supported by the DICE framework will be assessed in terms of how it handles such data requirements: shared data structures and diverse data layouts. The usability of the DICE memory model with regard to supporting the specific requirements to these lighting algorithms will be qualitatively assessed through the additional effort required from the programmer.

One of the main challenges present when addressing heterogeneous systems is the management of multiple implementations, one for each arquitecture. In order to minimize this difficulty, both implementations use most of the basic structures and use specialized ray tracing frameworks, which simplifies development. Another important aspecto to take into account is to minimize communication across devices, which can be costly.

Other problem specific challenges is load ballancing, which is an issue as whith Monte Carlo methods the workloads may be highly irregular, issue that should be addressed by the DICE scheduler.


A sintese de imagens fotorrealistas é um desafio ainda por resolver, dado que existem muitos algoritmos de transporte de luz eficientes, porem nenhum lida com todas as situações de forma robusta. O desempenho destas técnicas pode ser melhorado usando algoritmos de transporte de luz mais eficientes e robustos bem como usando todos os recursos computacionais da melhor forma possivel. Apesar da maioria dos computadores modernos ter disponível poderosos CPU's \textit{multicore} bem como GPU's \textit{manycore}, a maioria das implementações apenas utiliza um dos tipos de unidades computacionais. Este trabalho tem como alvo estas falhas de eficiencia avaliando os mais recentes algoritmos de iluminação global baseados em \textit{path space integration} usando a \textit{framework} DICE para sistemas heterogéneos (CPU + GPU). Os resultados de escalabilidade neste tipo de plataformas heterogéneas mostram resultados promissores e um caminho que vale a pena explorar.

A síntese de imagens fotorrealistas eficiente é ainda um desafio real devido à variedade de algoritmos disponíveis, bem como às exigências computacionais destes tipos de algoritmos. O desempenho destas técnicas pode ser melhorado usando duas abordagens complementares: utilizar os algoritmos de transporte de luz mais eficientes e robustos bem como usando todos os recursos computacionais da melhor forma possível. Apesar da maioria dos computadores modernos ter disponível poderosos CPU's \textit{multicore} bem como GPU's \textit{manycore}, a maioria das implementações apenas utiliza um dos tipos de unidades computacionais, não usando eficientemente todos os recursos disponíveis. Este trabalho tem como alvo estas falhas de eficiência avaliando os mais recentes algoritmos de iluminação global baseados em \textit{path space integration}, e avaliando a uma \textit{framework} que automaticamente tira total partido de sistemas computacionais heterogéneos: a \textit{framework} DICE. Os resultados de escalabilidade neste tipo de plataformas heterogéneas mostram resultados promissores e um caminho que vale a pena explorar.
